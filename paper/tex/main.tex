%% bare_jrnl_compsoc.tex
%% V1.4b
%% 2015/08/26
%% by Michael Shell
%% See:
%% http://www.michaelshell.org/
%% for current contact information.
%%
%% This is a skeleton file demonstrating the use of IEEEtran.cls
%% (requires IEEEtran.cls version 1.8b or later) with an IEEE
%% Computer Society journal paper.
%%
%% Support sites:
%% http://www.michaelshell.org/tex/ieeetran/
%% http://www.ctan.org/pkg/ieeetran
%% and
%% http://www.ieee.org/

%%*************************************************************************
%% Legal Notice:
%% This code is offered as-is without any warranty either expressed or
%% implied; without even the implied warranty of MERCHANTABILITY or
%% FITNESS FOR A PARTICULAR PURPOSE!
%% User assumes all risk.
%% In no event shall the IEEE or any contributor to this code be liable for
%% any damages or losses, including, but not limited to, incidental,
%% consequential, or any other damages, resulting from the use or misuse
%% of any information contained here.
%%
%% All comments are the opinions of their respective authors and are not
%% necessarily endorsed by the IEEE.
%%
%% This work is distributed under the LaTeX Project Public License (LPPL)
%% ( http://www.latex-project.org/ ) version 1.3, and may be freely used,
%% distributed and modified. A copy of the LPPL, version 1.3, is included
%% in the base LaTeX documentation of all distributions of LaTeX released
%% 2003/12/01 or later.
%% Retain all contribution notices and credits.
%% ** Modified files should be clearly indicated as such, including  **
%% ** renaming them and changing author support contact information. **
%%*************************************************************************


% *** Authors should verify (and, if needed, correct) their LaTeX system  ***
% *** with the testflow diagnostic prior to trusting their LaTeX platform ***
% *** with production work. The IEEE's font choices and paper sizes can   ***
% *** trigger bugs that do not appear when using other class files.       ***                          ***
% The testflow support page is at:
% http://www.michaelshell.org/tex/testflow/


\documentclass[10pt,journal,compsoc]{IEEEtran}
%
% If IEEEtran.cls has not been installed into the LaTeX system files,
% manually specify the path to it like:
% \documentclass[10pt,journal,compsoc]{../sty/IEEEtran}

\usepackage[ngerman]{babel}
\usepackage[utf8]{inputenc}
\usepackage[T1]{fontenc}
\usepackage{graphicx}
\usepackage{listings}

\graphicspath{{resources/}}

% Some very useful LaTeX packages include:
% (uncomment the ones you want to load)


% *** MISC UTILITY PACKAGES ***
%
%\usepackage{ifpdf}
% Heiko Oberdiek's ifpdf.sty is very useful if you need conditional
% compilation based on whether the output is pdf or dvi.
% usage:
% \ifpdf
%   % pdf code
% \else
%   % dvi code
% \fi
% The latest version of ifpdf.sty can be obtained from:
% http://www.ctan.org/pkg/ifpdf
% Also, note that IEEEtran.cls V1.7 and later provides a builtin
% \ifCLASSINFOpdf conditional that works the same way.
% When switching from latex to pdflatex and vice-versa, the compiler may
% have to be run twice to clear warning/error messages.






% *** CITATION PACKAGES ***
%
% \ifCLASSOPTIONcompsoc
%   % IEEE Computer Society needs nocompress option
%   % requires cite.sty v4.0 or later (November 2003)
%   \usepackage[nocompress]{cite}
% \else
%   % normal IEEE
%   \usepackage{cite}
% \fi
% cite.sty was written by Donald Arseneau
% V1.6 and later of IEEEtran pre-defines the format of the cite.sty package
% \cite{} output to follow that of the IEEE. Loading the cite package will
% result in citation numbers being automatically sorted and properly
% "compressed/ranged". e.g., [1], [9], [2], [7], [5], [6] without using
% cite.sty will become [1], [2], [5]--[7], [9] using cite.sty. cite.sty's
% \cite will automatically add leading space, if needed. Use cite.sty's
% noadjust option (cite.sty V3.8 and later) if you want to turn this off
% such as if a citation ever needs to be enclosed in parenthesis.
% cite.sty is already installed on most LaTeX systems. Be sure and use
% version 5.0 (2009-03-20) and later if using hyperref.sty.
% The latest version can be obtained at:
% http://www.ctan.org/pkg/cite
% The documentation is contained in the cite.sty file itself.
%
% Note that some packages require special options to format as the Computer
% Society requires. In particular, Computer Society  papers do not use
% compressed citation ranges as is done in typical IEEE papers
% (e.g., [1]-[4]). Instead, they list every citation separately in order
% (e.g., [1], [2], [3], [4]). To get the latter we need to load the cite
% package with the nocompress option which is supported by cite.sty v4.0
% and later. Note also the use of a CLASSOPTION conditional provided by
% IEEEtran.cls V1.7 and later.





% *** GRAPHICS RELATED PACKAGES ***
%
% \ifCLASSINFOpdf
  % \usepackage[pdftex]{graphicx}
  % declare the path(s) where your graphic files are
  % \graphicspath{{../pdf/}{../jpeg/}}
  % and their extensions so you won't have to specify these with
  % every instance of \includegraphics
  % \DeclareGraphicsExtensions{.pdf,.jpeg,.png}
% \else
  % or other class option (dvipsone, dvipdf, if not using dvips). graphicx
  % will default to the driver specified in the system graphics.cfg if no
  % driver is specified.
  % \usepackage[dvips]{graphicx}
  % declare the path(s) where your graphic files are
  % \graphicspath{{../eps/}}
  % and their extensions so you won't have to specify these with
  % every instance of \includegraphics
  % \DeclareGraphicsExtensions{.eps}
% \fi
% graphicx was written by David Carlisle and Sebastian Rahtz. It is
% required if you want graphics, photos, etc. graphicx.sty is already
% installed on most LaTeX systems. The latest version and documentation
% can be obtained at:
% http://www.ctan.org/pkg/graphicx
% Another good source of documentation is "Using Imported Graphics in
% LaTeX2e" by Keith Reckdahl which can be found at:
% http://www.ctan.org/pkg/epslatex
%
% latex, and pdflatex in dvi mode, support graphics in encapsulated
% postscript (.eps) format. pdflatex in pdf mode supports graphics
% in .pdf, .jpeg, .png and .mps (metapost) formats. Users should ensure
% that all non-photo figures use a vector format (.eps, .pdf, .mps) and
% not a bitmapped formats (.jpeg, .png). The IEEE frowns on bitmapped formats
% which can result in "jaggedy"/blurry rendering of lines and letters as
% well as large increases in file sizes.
%
% You can find documentation about the pdfTeX application at:
% http://www.tug.org/applications/pdftex






% *** MATH PACKAGES ***
%
%\usepackage{amsmath}
% A popular package from the American Mathematical Society that provides
% many useful and powerful commands for dealing with mathematics.
%
% Note that the amsmath package sets \interdisplaylinepenalty to 10000
% thus preventing page breaks from occurring within multiline equations. Use:
%\interdisplaylinepenalty=2500
% after loading amsmath to restore such page breaks as IEEEtran.cls normally
% does. amsmath.sty is already installed on most LaTeX systems. The latest
% version and documentation can be obtained at:
% http://www.ctan.org/pkg/amsmath





% *** SPECIALIZED LIST PACKAGES ***
%
%\usepackage{algorithmic}
% algorithmic.sty was written by Peter Williams and Rogerio Brito.
% This package provides an algorithmic environment fo describing algorithms.
% You can use the algorithmic environment in-text or within a figure
% environment to provide for a floating algorithm. Do NOT use the algorithm
% floating environment provided by algorithm.sty (by the same authors) or
% algorithm2e.sty (by Christophe Fiorio) as the IEEE does not use dedicated
% algorithm float types and packages that provide these will not provide
% correct IEEE style captions. The latest version and documentation of
% algorithmic.sty can be obtained at:
% http://www.ctan.org/pkg/algorithms
% Also of interest may be the (relatively newer and more customizable)
% algorithmicx.sty package by Szasz Janos:
% http://www.ctan.org/pkg/algorithmicx




% *** ALIGNMENT PACKAGES ***
%
%\usepackage{array}
% Frank Mittelbach's and David Carlisle's array.sty patches and improves
% the standard LaTeX2e array and tabular environments to provide better
% appearance and additional user controls. As the default LaTeX2e table
% generation code is lacking to the point of almost being broken with
% respect to the quality of the end results, all users are strongly
% advised to use an enhanced (at the very least that provided by array.sty)
% set of table tools. array.sty is already installed on most systems. The
% latest version and documentation can be obtained at:
% http://www.ctan.org/pkg/array


% IEEEtran contains the IEEEeqnarray family of commands that can be used to
% generate multiline equations as well as matrices, tables, etc., of high
% quality.




% *** SUBFIGURE PACKAGES ***
%\ifCLASSOPTIONcompsoc
%  \usepackage[caption=false,font=footnotesize,labelfont=sf,textfont=sf]{subfig}
%\else
%  \usepackage[caption=false,font=footnotesize]{subfig}
%\fi
% subfig.sty, written by Steven Douglas Cochran, is the modern replacement
% for subfigure.sty, the latter of which is no longer maintained and is
% incompatible with some LaTeX packages including fixltx2e. However,
% subfig.sty requires and automatically loads Axel Sommerfeldt's caption.sty
% which will override IEEEtran.cls' handling of captions and this will result
% in non-IEEE style figure/table captions. To prevent this problem, be sure
% and invoke subfig.sty's "caption=false" package option (available since
% subfig.sty version 1.3, 2005/06/28) as this is will preserve IEEEtran.cls
% handling of captions.
% Note that the Computer Society format requires a sans serif font rather
% than the serif font used in traditional IEEE formatting and thus the need
% to invoke different subfig.sty package options depending on whether
% compsoc mode has been enabled.
%
% The latest version and documentation of subfig.sty can be obtained at:
% http://www.ctan.org/pkg/subfig




% *** FLOAT PACKAGES ***
%
%\usepackage{fixltx2e}
% fixltx2e, the successor to the earlier fix2col.sty, was written by
% Frank Mittelbach and David Carlisle. This package corrects a few problems
% in the LaTeX2e kernel, the most notable of which is that in current
% LaTeX2e releases, the ordering of single and double column floats is not
% guaranteed to be preserved. Thus, an unpatched LaTeX2e can allow a
% single column figure to be placed prior to an earlier double column
% figure.
% Be aware that LaTeX2e kernels dated 2015 and later have fixltx2e.sty's
% corrections already built into the system in which case a warning will
% be issued if an attempt is made to load fixltx2e.sty as it is no longer
% needed.
% The latest version and documentation can be found at:
% http://www.ctan.org/pkg/fixltx2e


%\usepackage{stfloats}
% stfloats.sty was written by Sigitas Tolusis. This package gives LaTeX2e
% the ability to do double column floats at the bottom of the page as well
% as the top. (e.g., "\begin{figure*}[!b]" is not normally possible in
% LaTeX2e). It also provides a command:
%\fnbelowfloat
% to enable the placement of footnotes below bottom floats (the standard
% LaTeX2e kernel puts them above bottom floats). This is an invasive package
% which rewrites many portions of the LaTeX2e float routines. It may not work
% with other packages that modify the LaTeX2e float routines. The latest
% version and documentation can be obtained at:
% http://www.ctan.org/pkg/stfloats
% Do not use the stfloats baselinefloat ability as the IEEE does not allow
% \baselineskip to stretch. Authors submitting work to the IEEE should note
% that the IEEE rarely uses double column equations and that authors should try
% to avoid such use. Do not be tempted to use the cuted.sty or midfloat.sty
% packages (also by Sigitas Tolusis) as the IEEE does not format its papers in
% such ways.
% Do not attempt to use stfloats with fixltx2e as they are incompatible.
% Instead, use Morten Hogholm'a dblfloatfix which combines the features
% of both fixltx2e and stfloats:
%
% \usepackage{dblfloatfix}
% The latest version can be found at:
% http://www.ctan.org/pkg/dblfloatfix




%\ifCLASSOPTIONcaptionsoff
%  \usepackage[nomarkers]{endfloat}
% \let\MYoriglatexcaption\caption
% \renewcommand{\caption}[2][\relax]{\MYoriglatexcaption[#2]{#2}}
%\fi
% endfloat.sty was written by James Darrell McCauley, Jeff Goldberg and
% Axel Sommerfeldt. This package may be useful when used in conjunction with
% IEEEtran.cls'  captionsoff option. Some IEEE journals/societies require that
% submissions have lists of figures/tables at the end of the paper and that
% figures/tables without any captions are placed on a page by themselves at
% the end of the document. If needed, the draftcls IEEEtran class option or
% \CLASSINPUTbaselinestretch interface can be used to increase the line
% spacing as well. Be sure and use the nomarkers option of endfloat to
% prevent endfloat from "marking" where the figures would have been placed
% in the text. The two hack lines of code above are a slight modification of
% that suggested by in the endfloat docs (section 8.4.1) to ensure that
% the full captions always appear in the list of figures/tables - even if
% the user used the short optional argument of \caption[]{}.
% IEEE papers do not typically make use of \caption[]'s optional argument,
% so this should not be an issue. A similar trick can be used to disable
% captions of packages such as subfig.sty that lack options to turn off
% the subcaptions:
% For subfig.sty:
% \let\MYorigsubfloat\subfloat
% \renewcommand{\subfloat}[2][\relax]{\MYorigsubfloat[]{#2}}
% However, the above trick will not work if both optional arguments of
% the \subfloat command are used. Furthermore, there needs to be a
% description of each subfigure *somewhere* and endfloat does not add
% subfigure captions to its list of figures. Thus, the best approach is to
% avoid the use of subfigure captions (many IEEE journals avoid them anyway)
% and instead reference/explain all the subfigures within the main caption.
% The latest version of endfloat.sty and its documentation can obtained at:
% http://www.ctan.org/pkg/endfloat
%
% The IEEEtran \ifCLASSOPTIONcaptionsoff conditional can also be used
% later in the document, say, to conditionally put the References on a
% page by themselves.




% *** PDF, URL AND HYPERLINK PACKAGES ***
%
%\usepackage{url}
% url.sty was written by Donald Arseneau. It provides better support for
% handling and breaking URLs. url.sty is already installed on most LaTeX
% systems. The latest version and documentation can be obtained at:
% http://www.ctan.org/pkg/url
% Basically, \url{my_url_here}.





% *** Do not adjust lengths that control margins, column widths, etc. ***
% *** Do not use packages that alter fonts (such as pslatex).         ***
% There should be no need to do such things with IEEEtran.cls V1.6 and later.
% (Unless specifically asked to do so by the journal or conference you plan
% to submit to, of course. )


% correct bad hyphenation here
\hyphenation{op-tical net-works semi-conduc-tor}


\begin{document}
%
% paper title
% Titles are generally capitalized except for words such as a, an, and, as,
% at, but, by, for, in, nor, of, on, or, the, to and up, which are usually
% not capitalized unless they are the first or last word of the title.
% Linebreaks \\ can be used within to get better formatting as desired.
% Do not put math or special symbols in the title.
\title{Funktionale Programmierung -- \\Nebenläufigkeit \& Parallelisierung}
%
%
% author names and IEEE memberships
% note positions of commas and nonbreaking spaces ( ~ ) LaTeX will not break
% a structure at a ~ so this keeps an author's name from being broken across
% two lines.
% use \thanks{} to gain access to the first footnote area
% a separate \thanks must be used for each paragraph as LaTeX2e's \thanks
% was not built to handle multiple paragraphs
%
%
%\IEEEcompsocitemizethanks is a special \thanks that produces the bulleted
% lists the Computer Society journals use for "first footnote" author
% affiliations. Use \IEEEcompsocthanksitem which works much like \item
% for each affiliation group. When not in compsoc mode,
% \IEEEcompsocitemizethanks becomes like \thanks and
% \IEEEcompsocthanksitem becomes a line break with idention. This
% facilitates dual compilation, although admittedly the differences in the
% desired content of \author between the different types of papers makes a
% one-size-fits-all approach a daunting prospect. For instance, compsoc
% journal papers have the author affiliations above the "Manuscript
% received ..."  text while in non-compsoc journals this is reversed. Sigh.

\author{Jan-Philipp~Willem,~\IEEEmembership{1314162, Fakultät für Informatik, Hochschule Mannheim}
% \IEEEcompsocitemizethanks{\IEEEcompsocthanksitem M. Shell was with the Department
% of Electrical and Computer Engineering, Georgia Institute of Technology, Atlanta,
% GA, 30332.\protect\\
% % note need leading \protect in front of \\ to get a newline within \thanks as
% \\ is fragile and will error, could use \hfil\break instead.
% E-mail: see http://www.michaelshell.org/contact.html
% \IEEEcompsocthanksitem J. Doe and J. Doe are with Anonymous University.}% <-this % stops an unwanted space
% \thanks{Ich möchte mich bei Prof. Dr. Sandro Leuchter für seine Betreung bedanken.}
}

% note the % following the last \IEEEmembership and also \thanks -
% these prevent an unwanted space from occurring between the last author name
% and the end of the author line. i.e., if you had this:
%
% \author{....lastname \thanks{...} \thanks{...} }
%                     ^------------^------------^----Do not want these spaces!
%
% a space would be appended to the last name and could cause every name on that
% line to be shifted left slightly. This is one of those "LaTeX things". For
% instance, "\textbf{A} \textbf{B}" will typeset as "A B" not "AB". To get
% "AB" then you have to do: "\textbf{A}\textbf{B}"
% \thanks is no different in this regard, so shield the last } of each \thanks
% that ends a line with a % and do not let a space in before the next \thanks.
% Spaces after \IEEEmembership other than the last one are OK (and needed) as
% you are supposed to have spaces between the names. For what it is worth,
% this is a minor point as most people would not even notice if the said evil
% space somehow managed to creep in.



% The paper headers
\markboth{Funktionale Programmierung -- Nebenläufigkeit \& Parallelisierung, Seminar WS2016}%
{}
% The only time the second header will appear is for the odd numbered pages
% after the title page when using the twoside option.
%
% *** Note that you probably will NOT want to include the author's ***
% *** name in the headers of peer review papers.                   ***
% You can use \ifCLASSOPTIONpeerreview for conditional compilation here if
% you desire.



% The publisher's ID mark at the bottom of the page is less important with
% Computer Society journal papers as those publications place the marks
% outside of the main text columns and, therefore, unlike regular IEEE
% journals, the available text space is not reduced by their presence.
% If you want to put a publisher's ID mark on the page you can do it like
% this:
%\IEEEpubid{0000--0000/00\$00.00~\copyright~2015 IEEE}
% or like this to get the Computer Society new two part style.
%\IEEEpubid{\makebox[\columnwidth]{\hfill 0000--0000/00/\$00.00~\copyright~2015 IEEE}%
%\hspace{\columnsep}\makebox[\columnwidth]{Published by the IEEE Computer Society\hfill}}
% Remember, if you use this you must call \IEEEpubidadjcol in the second
% column for its text to clear the IEEEpubid mark (Computer Society jorunal
% papers don't need this extra clearance.)



% use for special paper notices
%\IEEEspecialpapernotice{(Invited Paper)}



% for Computer Society papers, we must declare the abstract and index terms
% PRIOR to the title within the \IEEEtitleabstractindextext IEEEtran
% command as these need to go into the title area created by \maketitle.
% As a general rule, do not put math, special symbols or citations
% in the abstract or keywords.
\IEEEtitleabstractindextext{%
\begin{abstract}
Durch die immer weiter fortschreitende Vernetzung und die damit verbundene Fokussierung auf Big-Data, steigen auch die Erwartungen der User an produktiv genutzten Systemen.
Um diesen Anforderungen gerecht zu werden, wird es immer wichtiger, effizient und nebenläufig programmieren zu können.
Die bisher eingesetzten Lösungen haben durch den Einsatz imperativer Vorgehensweisen hohe Fehleranfälligkeit und Schwächen bei der Entwicklung von Multi-Tasking-Anwendungen bewiesen.
Diese Arbeit hat sich das Ziel gesetzt, einige Konzepte der Nebenläufigkeit und Parallelisierung in der Funktionalen Programmierung an Hand der Sprache Elixir zu demonstrieren. \\
~\\
\textbf{Abstract}---The increasing Interconnection and the related Big-Data play an integrative part in the advancing expectations in productive-used systems by its users.
To fulfill these requirements, the need to program efficiently and concurrent will be even more important.
When building Multi-Tasking-Systems, previously used solutions were highly error-prone and had definitive weaknesses, due to the use of imperative strategies.
This text strives to demonstrate some concepts of Concurrency and Parallelism with Functional-Programming in context of the language Elixir.
\end{abstract}

% Note that keywords are not normally used for peerreview papers.
% \begin{IEEEkeywords}
% Computer Society, IEEE, IEEEtran, journal, \LaTeX, paper, template.
% \end{IEEEkeywords}
}


% make the title area
\maketitle


% To allow for easy dual compilation without having to reenter the
% abstract/keywords data, the \IEEEtitleabstractindextext text will
% not be used in maketitle, but will appear (i.e., to be "transported")
% here as \IEEEdisplaynontitleabstractindextext when the compsoc
% or transmag modes are not selected <OR> if conference mode is selected
% - because all conference papers position the abstract like regular
% papers do.
% \IEEEdisplaynontitleabstractindextext
% \IEEEdisplaynontitleabstractindextext has no effect when using
% compsoc or transmag under a non-conference mode.



% For peer review papers, you can put extra information on the cover
% page as needed:
% \ifCLASSOPTIONpeerreview
% \begin{center} \bfseries EDICS Category: 3-BBND \end{center}
% \fi
%
% For peerreview papers, this IEEEtran command inserts a page break and
% creates the second title. It will be ignored for other modes.
% \IEEEpeerreviewmaketitle



\IEEEraisesectionheading{
  \section{Einleitung}\label{sec:einleitung}}

% Computer Society journal (but not conference!) papers do something unusual
% with the very first section heading (almost always called "Introduction").
% They place it ABOVE the main text! IEEEtran.cls does not automatically do
% this for you, but you can achieve this effect with the provided
% \IEEEraisesectionheading{} command. Note the need to keep any \label that
% is to refer to the section immediately after \section in the above as
% \IEEEraisesectionheading puts \section within a raised box.




% The very first letter is a 2 line initial drop letter followed
% by the rest of the first word in caps (small caps for compsoc).
%
% form to use if the first word consists of a single letter:
% \IEEEPARstart{A}{demo} file is ....
%
% form to use if you need the single drop letter followed by
% normal text (unknown if ever used by the IEEE):
% \IEEEPARstart{A}{}demo file is ....
%
% Some journals put the first two words in caps:
% \IEEEPARstart{T}{his demo} file is ....
%
% Here we have the typical use of a "T" for an initial drop letter
% and "HIS" in caps to complete the first word.


  \begin{quotation}
The complexity for minimum component costs has increased at a rate of roughly a factor of two per year. [...] Over the longer term, the rate of increase is a bit more uncertain, although there is no reason to believe it will not remain nearly constant for at least 10 years. (G. E. Moore, 1965)
  \end{quotation}
  ~\\
\IEEEPARstart{S}{eit} Gordon Moore im April 1965 seine Beobachtung als einen Artikel in der Zeitschrift ``Electronics`` veröffentlichte, ist viel Zeit vergangen.
Die auch als ``Moore's Law`` bekannte These beschreibt, dass sich jedes Jahr in Relation zu minimalen Komponentenkosten, die Integrationsdichte in Integrierten Schaltkreisen (englisch: integrated circuits, ICs) verdoppelt. 
\cite{mooreElectronics}
  
\begin{figure}
\centering
\includegraphics[width=0.5\textwidth]{moores_law_black.pdf}
\caption{Transistor counts 1971-2011 \& Moore's Law}
\label{fig_moores}
\end{figure}

In Abbildung \ref{fig_moores} ist eine Relation zwischen CPU-Produkt-Veröffentlichung und deren jeweilige Transistoren-Anzahl zu betrachten. Man sieht, dass sich die Komplexität der ICs verdoppelt hat. Deswegen wird vermutet, dass sich die Moore's Law bewahrheitet hat. Jedoch scheint dieses Verhalten bald ein Ende zu haben.
So hat Intel kürzlich mit dem Veröffenlichen der neuen Prozessorgeneration \textit{Kaby Lake}, einen Prozessor auf den Markt gebracht, welcher eine exakt gleiche Performance wie sein Vorgänger \textit{Skylake} aufweist.
% You must have at least 2 lines in the paragraph with the drop letter
% (should never be an issue)


% needed in second column of first page if using \IEEEpubid
%\IEEEpubidadjcol

\subsection{Parallelisierung}
Der deutsche Begriff \textit{parallel} besitzt die Synonyme \textit{nebeneinander} und \textit{nebenläufig}. Dies trifft jedoch im Kontext der Informatik nicht zu. Die Parallelisierung hat zwar Gemeinsamkeiten mit der Nebenläufigkeit, allerdings auch Unterschiede.
\\Man spricht auch von \textit{Multi-Processing-Systemen}, welche einen Geschwindigkeitsvorteil durch gleichzeitiges Ausführen von \textit{Anweisungen}, gegenüber eines sequenziellen Systems erzielen. \cite{sevenCon} 
Nach einer Aufteilung eines Problems in Teilprobleme, werden diese auf verfügbaren Prozessor-Kernen \textit{gleichzeitig} gelöst.

\subsection{Nebenläufigkeit}
Die im Englischen als ``Concurrency`` bezeichnete Nebenläufigkeit, beschäftigt sich dahingegen um das Ausführen von mehreren \textit{Aufgaben} zur gleichen Zeit. \cite{sevenCon}
\\Der Fokus liegt hierin an der Betrachtung der Architektur hinter dem System, weniger auf deren internen Aufbau. Die Einzelnen Komponenten selbst, müssen nicht zwingend eine Parallelisierung durchführen. Durch eine starke und vor allem sinnvolle Kapselung innerhalb dieser \textit{Multi-Tasking-Systeme} kann jedoch in den meisten Fällen eine Parallelisierung auf trivialer Weise vorgenommen werden. Teilen sich die einzelnen Komponenten des Systems referenzierten Zustand gegenseitig, so sollte auf Unveränderlichkeit der enthaltenen Daten geachtet werden.
\subsection{„Concurrency is not Parallelism“}
Rob Pike, ein Entwickler der Go-Lang bei Google hat die Unterschiede von Nebenläufigkeit und Parallelisierung in einem Talk auf der ``Heroku's Waza conference`` im Januar 2012 sehr interessant zusammengefasst..

\quote{
  - Concurrency is about dealing with lots of things at once.
  \\ - Parallelism is about doing lots of things at once.
  \\ - Concurrency is about structure, parallelism is about execution.
  \\(Rob Pike, 2012)
}
~\\

\section{Funktionales Paradigma}

\subsection{Unveränderlichkeit}

\begin{lstlisting}[frame=single]
a = 3     a = 3
a += 2    a' = add(a, 2)
a -> 5    a' -> 5
\end{lstlisting}
\small{Listing 1. Mutability vs. Immutability}\\
~\\
Viele Fehler traten in der Vergangenheit durch die Veränderung von bestehendem Zustand auf, welcher typischerweise in verschiedenen Bereichen der Applikation referenziert wurde. Die Ergebnisse sind meist von unerwarteter Natur. Das Prinzip der Unveränderlichkeit beschreibt die Vorgehensweise, bei nötigen Zustandsänderungen stattdessen eine veränderte Kopie zurückzuliefern.
Falls gewünscht, so kann man diese Zustandskopien wie Snapshots in einem Cache betrachten und sich beliebig schrittweise zwischen diesen bewegen. 

Listing 1 zeigt eine typische Vorgehensweise beim Zuweisen von Variablen. Der Variable \texttt{a} wird jeweils die 3 zugewiesen. Imperativ würde man anschließend den Wert verändern um die gewünschte Funktionalität zu erreichen. Im Gegenzug dazu würde der funktionale Ansatz vorsehen, das Ergebnis der Addition einer neuen Variable zuzuweisen als den alten Zustand zu verändern.
\subsection{Funktionen}
Die Funktionale Programmierung fokussiert sich in erster Linie auf Funktions-Komposition, bei der komplexe Schachtelungen von Teilfunktionen miteinander vereint werden. Dabei wird jedoch sonst auf gebräuchliche Seiten-Effekte verzichtet. Man möchte so ein möglichst deterministisches Verhalten bewerkstelligen. Somit sollte eine Funktion auch immer nur die eine Sache tun, welche man beim Lesen der Funktionssignatur auch erwarten würde. Im Idealfall können die einzelnen Funktionen als reine Daten-Transformationen aufgefasst werden, welche die übergebenen Daten kopieren, transformieren und wieder zurückgeben. Dies wird in der Literatur auf verschiedene Arten beschrieben: ``pure`` und ``data-in, data-out`` sind Synonyme für einander. In der Informatik ist dieses Prinzip auch als Eingabe-Verarbeitung-Ausgabe (EVA) fest verankert. \\
Funktionen können selbst auch als ein Datentyp definiert werden. Das sogenannte Konzept ``functions as first-class citizens``, ermöglicht es Funktionen als Parameter anderer Funktionen zu übergeben. Die Ausführende Funktion kann jedoch selbst entscheiden, ob die übergebene Funktion schon ausgewertet werden soll. (lazyness)\\
Lamdas sind anonyme Funktionen, welche inline definiert werden und somit keinen Namen besitzen. Sie werden oft als Callback anderer Funktionen genutzt oder für die spätere Weiterverwendung einer Variable zugewiesen.

\subsection{reine funktionale Sprachen}
Es wird zwischen Sprachen unterscheiden, welche eine Unterstützung für die funktionale Programmierung bieten und solchen, welche ausschließlich auf deren Konzepte beruhen.
Seiten-Effekte sind somit laut Definition gar nicht möglich auszuführen. Dies kann sehr hilfreich sein, da in der Regel auch solche Teile einer Programmiersprache genutzt werden, welche man eigentlich meiden sollte. Gerade bei der Arbeit in einem größeren Team, stellen sich die zusätzlichen Einschränkungen eher als eine Erleichterung heraus.\\
Viele rein funktionale Sprachen wie beispielsweise \textit{Haskell} setzen zudem eine strikte Typisierung voraus. Im Gegensatz zu beispielsweise \textit{Java} hat der Programmierer einen tatsächlichen Nutzen von der Typisierung, da ihm so der Compiler beim beseitigen von Fehlern besser helfen kann.
\section{Elixir}
\begin{figure}
\centering
\includegraphics[width=0.3\textwidth]{supervision.pdf}
\caption{komplexe Supervision-Bäume}
\label{fig_supervision}
\end{figure}
Elixir \cite{programmingElixir} ist eine seit dem Jahre 2011 entwickelte Variante von Erlang. Es besteht eine dynamische Typisierung mit Typen-Interferenz jedoch weißt sie weiterhin Features einer reinen funktionalen Sprache auf. Es kann weitestgehend Seiten-Effekt-Frei programmiert werden. Elixir nutzt Erlangs Beam-VM um eine Fehler-Tolerante Umgebung zu bieten. Dazu wird mit einem Kompiler ein Erlang-Bytecode erstellt.
\\ Erlang wurde von Ericsson schon im Jahre 1987 entwickelt, um die Vernetzung von Telefonie-Systemen zu vereinfachen.
Grundlage sind sogenannte Akteure, welche aus leichtgewichtigen Erlang-Prozessen bestehen. Diese erzeugen und nutzen wiederum unzählige Prozesse um ihre Aufgaben zu lösen. Es sind je nach Anwendungsfall sowohl geteilte wie auch verteilte Speicher zwischen den Akteuren möglich.
Im Rahmen des \textit{Open-Telephony-Protocol (OTP)} ist es üblich komplexe Akteur-Bäume aufzubauen. (Abbildung \ref{fig_supervision}) Ein Akteur kann so auch als ein Supervisor eines anderen Prozesses dienen. Die Fehlerbehandlung kann damit erheblich vereinfacht werden. Es besteht die Philosophie ``let it crash``, da man bei einem Fehlerfall, den Prozess einfach neu starten kann. Es existieren viele Arten an Vorgehensweisen, welche Teile der Applikation vom neu starten betroffen sind. Wird das Konzept der sinnvollen Aufteilung in Akteure verfolgt, so erhält man granulare Teilsysteme, die sich nicht beeinflussen können.
\subsection{OTP / Actor-Model}
Das Actor-Modell stellt das Concurrency-Model in Elixir dar. Es wurde oft implementiert, jedoch trug Erlang den weitesten Nutzen daraus, da es weitestgehend auf dessen Konzepten basiert.
\\ Die Grundidee besteht aus unabhängigen Akteuren, welche jeweils eine eigene Mailbox und einen eigenen Zustand besitzen. Die Akteure können untereinander Nachrichten austauschen um miteinander zu kommunizieren. Die Reihenfolge der Abarbeitung der Nachrichten ist nach deren Eintreffen geregelt. Je nach Anwendungsfall, kann es sich anbieten die Akteure als vollwertige eigene Services zu betrachten und sie mit einem eigenen Cache und oder Persistenz (DB) auszustatten.
\\ Als Alternativen gibt es Libraries, mit denen man mit Akteuren in beliebigen Sprachen programmieren kann. So gibt es beispielsweise Implementierungen für Java/Scala (Akka), Python (Pykka), Akka.NET, C++ (CAF) und Ruby (Celluloid).
\subsection{List-Processing: map}

\begin{figure}[!h]
\centering
\includegraphics[width=0.3\textwidth]{map.pdf}
\caption{Die \texttt{map}-Funktion}
\label{fig_map}
\end{figure}
Eines der wichtigsten Funktionen einer funktionalen Sprache, ist die \texttt{map}-Funktion. Sie ermöglicht eine schrittweise transformation einer Datenstruktur. Als Paramter werden in Elixir eine \texttt{transform}-Funktion und eine Collection übergeben. Beim Auswerten, wird über die Collection interiert, das jeweilige Element verändert und wieder einer neuen Collection hinzugefügt. Der Rückgabewert kann jedoch auch von einem anderen Typ sein.
\\
\begin{lstlisting}
iex> Enum.map [1, 2, 3], fn x -> x + 1 end
  [2, 3, 4]
iex> Enum.map [1, 2, 3], &(&1 * &1)
  [1, 4, 9]
iex> defmodule Math do
...> def multWithKey({k, v}), do: k * v
...> end
  ...
iex> list = Enum.with_index([1, 2, 3])
  [{1, 0}, {2, 1}, {3, 2}]
iex> Enum.map list, &Math.multWithKey/1
  [0, 2, 6]
\end{lstlisting}
\small{Listing 2. Typische Einsatzmöglichkeiten von \texttt{map}}\\
~\\

\subsection{Streams}
 Mithilfe von Elixir Streams kann eine große (oder unendliche) Datenstruktur lazy verarbeitet werden. Dabei werden viele Funktionen aus dem \texttt{Enum}-Modul implementiert, so dass man ähnliche Verhalten abbilden kann. Mit function-composition wird die Collection nur einmal durchlaufen und dabei jeweilig transformiert. Sollten nicht alle Elemente gebraucht werden, so kann man mit \texttt{Enum.take} die Größe der Ergebnismenge bestimmt werden.\\
Das ganze Konzept stellt eine wichtige Optimierung unter Anderem bei der Parallelisierung dar. Im Detail gehen die Elixir-Streams und andere ähnliche Implementierungen in anderen Sprachen auf die Clojure Transducer/Reducer zurück.
\begin{lstlisting}
iex> 1..10000
     |> Stream.map(&(&1 * &1))
     |> Stream.map(&(&1 + &1))
     |> Stream.map(&IO.inspect(&1))
     |> Enum.take(10)
  2
  8
  18
  ..
  200
  [2, 8, 18, 32, 50, 72, 98, 128, 162, 200]
\end{lstlisting}
\small{Listing 3. Elixir-Streams}\\
~\\

\subsection{Processes}
Im Gegensatz zu der fehleranfälligen Thread-Programmierung in anderen Sprachen kann in Elixir mit \texttt{spawn} direkt ein leichtgewichtiger Prozess gestartet werden. Der Rückgabewert ist dabei eine Process-Id. Mit \texttt{send} ist es möglich zwischen den Prozessen Nachrichten auszutauschen. Und mit einem \texttt{receive}-Block können die eingegangenen Nachrichten verarbeitet werden.\\
\begin{lstlisting}
defmodule Parallel do
  def pmap(collection, fun) do
    me = self
    collection
    |> Enum.map(fn (elem) ->
         spawn fn -> 
           (send me,  self, fun.(elem) )
         end
       end)
    |> Enum.map(fn (pid) ->
         receive do
           ^pid, result  -> result
         end
       end)
  end
end

Parallel.pmap 1..1000, &(&1 * &1)
\end{lstlisting}
\small{Listing 4. Prozesse erzeugen}\\
~\\

\subsection{Tasks}
Trotz des sauberen Codes beim Erzeugen von Prozessen, ist diese Herangehensweise ebenso Fehleranfällig, da man sich um selbst über Supervisioning-Verhalten und beispielsweise Timeouts kümmern muss. In viele Fällen ist es deswegen sinnvoll, die von Elixir mitgebrachte OTP-Abstraktion der \texttt{Tasks} zu nutzen. Dadurch hat man den zusätzlichen Vorteil, dass man den Task in bestehende Supervision-Bäume integrieren kann. Gerade bei so einem häufigen Anwendungsfall eines Elixir-Prozesses wird einem einiges an Arbeit abgenommen und man muss sich jedoch trotzdem nicht mit dem zwar mächtigen aber teilweise auch komplexen OTP beschäftigen.\\
Die dabei erhaltene Lösung wirkt noch etwas sauberer zu lesen und es wurden schon einige Mögliche Fallstricke optimiert. So kann mit \texttt{Task.async} ein Task erzeugt werden, welcher die übergebene Funktion nicht-blockierend ausführt. Anschließend kann mit \texttt{Task.await} auf die Ergebnisse von Tasks gewartet werden.
\begin{lstlisting}
defmodule Parallel do
  def pmap(collection, func) do
    collection
    |> Enum.map(&(Task.async(fn -> func.(&1))))
    |> Enum.map(&Task.await/1)
  end
end

Parallel.pmap 1..1000, &(&1 * &1)
\end{lstlisting}
\small{Listing 5. Tasks erzeugen}\\
~\\

\section{Fazit}
Wenige Sprachen und deren Tool-Chains bieten dem Programmierer wirkliche Erleichterungen bei der Entwicklung von nebenläufigen Systemen. Oft fehlen weiterhin Konzepte die eine reine Parallelisierung ermöglichen. Die Thread-Programmierung ist dabei meist die einzigste Möglichkeit. \\
Je nach Sprache erinnert die Funktionale Programmierung eher an eine Zusammenstellung von Verhalten. Deren Art und Abfolge kann genau und deklarativ bestimmt werden. So beschäftigt sich \textit{Higher-Order-Programing} genau mit dieser Tatsache.
Der kompromisslose Einsatz von Unveränderlichkeit der Daten, erleichtert weiterhin die Vorgehensweise ein nebenläufiges oder paralleles System zu entwickeln.
Die typische Fehleranfälligkeit (Dead-Locks, Race-Conditions,..) bei der Synchronisierung einzelner Teilsysteme, wie es bei Threads mit geteiltem Zustand der Fall war, werden einfach umgangen.
\\Bei der Wahl der Sprache Elixir, erhält man eine direkte Unterstützung für eine Verteilung der Anwendung auf mehrere Knoten. Dies wird mit dem von Erlang etablierten OTP mithilfe der Beam-VM erzielt. Die Programmierung findet in einer Umgebung ohne Seiten-Effekten oder sich verändernden referenzierten Daten, ohne auf moderne Syntax oder Sprache-Features wie Meta-Programming verzichten zu müssen.
\\ Im Gegensatz zu den Vorteilen die geboten werden, steht die Tatsache, dass die Lernkurve um die funktionalen Konzepte zu verstehen, sehr steil sein kann.
Man muss die bisherig genutzten Methoden teilweise komplett aufgeben um anstelle davon neues zu lernen.
% An example of a floating figure using the graphicx package.
% Note that \label must occur AFTER (or within) \caption.
% For figures, \caption should occur after the \includegraphics.
% Note that IEEEtran v1.7 and later has special internal code that
% is designed to preserve the operation of \label within \caption
% even when the captionsoff option is in effect. However, because
% of issues like this, it may be the safest practice to put all your
% \label just after \caption rather than within \caption{}.
%
% Reminder: the "draftcls" or "draftclsnofoot", not "draft", class
% option should be used if it is desired that the figures are to be
% displayed while in draft mode.
%

% Note that the IEEE typically puts floats only at the top, even when this
% results in a large percentage of a column being occupied by floats.
% However, the Computer Society has been known to put floats at the bottom.


% An example of a double column floating figure using two subfigures.
% (The subfig.sty package must be loaded for this to work.)
% The subfigure \label commands are set within each subfloat command,
% and the \label for the overall figure must come after \caption.
% \hfil is used as a separator to get equal spacing.
% Watch out that the combined width of all the subfigures on a
% line do not exceed the text width or a line break will occur.
%
%\begin{figure*}[!t]
%\centering
%\subfloat[Case I]{\includegraphics[width=2.5in]{box}%
%\label{fig_first_case}}
%\hfil
%\subfloat[Case II]{\includegraphics[width=2.5in]{box}%
%\label{fig_second_case}}
%\caption{Simulation results for the network.}
%\label{fig_sim}
%\end{figure*}
%
% Note that often IEEE papers with subfigures do not employ subfigure
% captions (using the optional argument to \subfloat[]), but instead will
% reference/describe all of them (a), (b), etc., within the main caption.
% Be aware that for subfig.sty to generate the (a), (b), etc., subfigure
% labels, the optional argument to \subfloat must be present. If a
% subcaption is not desired, just leave its contents blank,
% e.g., \subfloat[].


% An example of a floating table. Note that, for IEEE style tables, the
% \caption command should come BEFORE the table and, given that table
% captions serve much like titles, are usually capitalized except for words
% such as a, an, and, as, at, but, by, for, in, nor, of, on, or, the, to
% and up, which are usually not capitalized unless they are the first or
% last word of the caption. Table text will default to \footnotesize as
% the IEEE normally uses this smaller font for tables.
% The \label must come after \caption as always.
%
%\begin{table}[!t]
%% increase table row spacing, adjust to taste
%\renewcommand{\arraystretch}{1.3}
% if using array.sty, it might be a good idea to tweak the value of
% \extrarowheight as needed to properly center the text within the cells
%\caption{An Example of a Table}
%\label{table_example}
%\centering
%% Some packages, such as MDW tools, offer better commands for making tables
%% than the plain LaTeX2e tabular which is used here.
%\begin{tabular}{|c||c|}
%\hline
%One & Two\\
%\hline
%Three & Four\\
%\hline
%\end{tabular}
%\end{table}


% Note that the IEEE does not put floats in the very first column
% - or typically anywhere on the first page for that matter. Also,
% in-text middle ("here") positioning is typically not used, but it
% is allowed and encouraged for Computer Society conferences (but
% not Computer Society journals). Most IEEE journals/conferences use
% top floats exclusively.
% Note that, LaTeX2e, unlike IEEE journals/conferences, places
% footnotes above bottom floats. This can be corrected via the
% \fnbelowfloat command of the stfloats package.








% if have a single appendix:
%\appendix[Proof of the Zonklar Equations]
% or
%\appendix  % for no appendix heading
% do not use \section anymore after \appendix, only \section*
% is possibly needed

% use appendices with more than one appendix
% then use \section to start each appendix
% you must declare a \section before using any
% \subsection or using \label (\appendices by itself
% starts a section numbered zero.)
%


% \appendices
% \section{Titel}
% Appendix one text goes here.

% you can choose not to have a title for an appendix
% if you want by leaving the argument blank
% \section{}
% Appendix two text goes here.

% % use section* for acknowledgment
% \ifCLASSOPTIONcompsoc
%   % The Computer Society usually uses the plural form
%   \section*{Acknowledgments}
% \else
%   % regular IEEE prefers the singular form
%   \section*{Acknowledgment}
% \fi
%
%
% The authors would like to thank...


% Can use something like this to put references on a page
% by themselves when using endfloat and the captionsoff option.
% \ifCLASSOPTIONcaptionsoff
  % \newpage
% \fi



% trigger a \newpage just before the given reference
% number - used to balance the columns on the last page
% adjust value as needed - may need to be readjusted if
% the document is modified later
%\IEEEtriggeratref{8}
% The "triggered" command can be changed if desired:
%\IEEEtriggercmd{\enlargethispage{-5in}}

% references section

% can use a bibliography generated by BibTeX as a .bbl file
% BibTeX documentation can be easily obtained at:
% http://mirror.ctan.org/biblio/bibtex/contrib/doc/
% The IEEEtran BibTeX style support page is at:
% http://www.michaelshell.org/tex/ieeetran/bibtex/
%\bibliographystyle{IEEEtran}
% argument is your BibTeX string definitions and bibliography database(s)
%\bibliography{IEEEabrv,../bib/paper}
%
% <OR> manually copy in the resultant .bbl file
% set second argument of \begin to the number of references
% (used to reserve space for the reference number labels box)
\begin{thebibliography}{1}

\bibitem{sevenCon}
  Paul Butcher, \emph{Seven Concurrency Models in Seven Weeks: When Threads Unravel}, 1.Auflage (18. Juli 2014), O'Reilly UK Ltd.

\bibitem{programmingElixir}
  Dave Thomas, \emph{Programming Elixir}, 1.Auflage (19. Oktober 2014), Pragmatic Programmers
  \bibitem{mooreElectronics}
  G. E. Moore, \emph{Cramming more components onto integrated circuits, Reprinted from Electronics, volume 38, number 8, April 19, 1965, pp.114 ff.,} in IEEE Solid-State Circuits Society Newsletter, vol. 11, no. 5, pp. 33-35, Sept. 2006.
\end{thebibliography}

% biography section
%
% If you have an EPS/PDF photo (graphicx package needed) extra braces are
% needed around the contents of the optional argument to biography to prevent
% the LaTeX parser from getting confused when it sees the complicated
% \includegraphics command within an optional argument. (You could create
% your own custom macro containing the \includegraphics command to make things
% simpler here.)
%\begin{IEEEbiography}[{\includegraphics[width=1in,height=1.25in,clip,keepaspectratio]{mshell}}]{Michael Shell}
% or if you just want to reserve a space for a photo:

% \begin{IEEEbiography}{Michael Shell}
% Biography text here.
% \end{IEEEbiography}

% if you will not have a photo at all:
% \begin{IEEEbiographynophoto}{John Doe}
% Biography text here.
% \end{IEEEbiographynophoto}

% insert where needed to balance the two columns on the last page with
% biographies
%\newpage


% You can push biographies down or up by placing
% a \vfill before or after them. The appropriate
% use of \vfill depends on what kind of text is
% on the last page and whether or not the columns
% are being equalized.

%\vfill

% Can be used to pull up biographies so that the bottom of the last one
% is flush with the other column.
%\enlargethispage{-5in}



% that's all folks
\end{document}


